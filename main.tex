\documentclass[12pt, a4paper]{article}

% --- PACOTES ---
\usepackage[utf8]{inputenc}
\usepackage[brazil]{babel}
\usepackage{geometry}
\usepackage{amsmath}
\usepackage{graphicx}
\usepackage{booktabs} % Para tabelas de alta qualidade
\usepackage{indentfirst} % Para indentar o primeiro parágrafo
\usepackage[alf]{abntex2cite} % Pacote ABNT para citações
\usepackage[hidelinks]{hyperref}

\geometry{a4paper, top=3cm, bottom=2cm, left=3cm, right=2cm} % Margens ABNT
\linespread{1.5} % Espaçamento 1.5 entre linhas

\begin{document}
\renewcommand{\bibname}{REFERÊNCIAS} % Renomeia a seção de bibliografia

% --- INÍCIO DA SEÇÃO DE METADADOS ---
\begin{titlepage}
    \centering
    \vspace*{0.5cm}
    \includegraphics[width=0.25\textwidth]{logo-ppgecam.jpeg}\\[0.2cm]
    \vspace*{0.5cm}
    \textbf{\small UNIVERSIDADE FEDERAL DA PARAÍBA \\
    CENTRO DE TECNOLOGIA \\
    PROGRAMA DE Pós-GRADUAÇÃO EM ENGENHARIA CIVIL E AMBIENTAL}
    \vspace{3cm}
    
    \textbf{\Huge Avaliação Multicritério do Risco Ambiental em Bacias Hidrográficas Urbanas com Sensoriamento Remoto e Google Earth Engine: Estudo de Caso do Rio Jaguaribe (PB)}
    \vspace{3cm}
    
    \textbf{\large Jaidna Dantas de Almeida}
    \vfill
    
    \textbf{JOÃO PESSOA, PB \\ \today}
\end{titlepage}

\begin{center}
    \textbf{Área de Concentração:} Recursos Hídricos e Saneamento Ambiental \\
    \textbf{Linha de Pesquisa:} Planejamento, gestão, monitoramento e modelos matemáticos em recursos hídricos \\
    \textbf{Orientador(a):} Prof. Dr. Gerald Norbert Souza da Silva \\
    \textbf{Palavras-chave:} sensoriamento remoto, índice de risco ambiental, plataforma web, Google Earth Engine, Rio Jaguaribe.
\end{center}
% --- FIM DA SEÇÃO DE METADADOS ---

\newpage

% --- INÍCIO DA SEÇÃO INTRODUÇÃO ---
\section*{1. INTRODUÇÃO}
A degradação de corpos hídricos urbanos é uma questão ambiental crítica no Brasil, exacerbada pelos desafios no monitoramento contínuo da qualidade da água \cite{santos2016}. Em resposta a essa lacuna, este projeto propõe uma metodologia escalável e automatizada para avaliar o risco ambiental em bacias urbanas, não apenas gerando dados, mas também democratizando seu acesso por meio de uma plataforma de monitoramento web.

Integrando o Google Earth Engine (GEE) com imagens de satélite e dados socioeconômicos, esta pesquisa desenvolve um fluxo de trabalho reprodutível para o cálculo de um Índice de Risco Ambiental (IRA). O objetivo é criar e implementar um sistema que forneça \textit{insights} sistemáticos e atualizados para a gestão ambiental, em conformidade com a legislação brasileira \cite{conama357, pnrh1997}.

Como prova de conceito, este estudo foca na Bacia do Rio Jaguaribe (PB), desenvolvendo uma ferramenta de apoio à decisão que traduz dados complexos de sensoriamento remoto em informações visuais e interativas, acessíveis a gestores e à sociedade.
% --- FIM DA SEÇÃO INTRODUÇÃO ---

% --- INÍCIO DA SEÇÃO OBJETIVOS (ATUALIZADA) ---
\section*{2. OBJETIVOS}
\subsection*{Objetivo Geral}
Desenvolver e implementar uma metodologia de avaliação de risco ambiental, baseada em análise multicritério e sensoriamento remoto, culminando em uma **plataforma web de monitoramento dinâmico** para a bacia do Rio Jaguaribe.

\subsection*{Objetivos Específicos}
\begin{itemize}
    \item Desenvolver um fluxo de trabalho automatizado no Google Earth Engine para estimar mensalmente parâmetros de qualidade da água (Clorofila-a, Sólidos Suspensos Totais).
    \item Integrar dados de pressão antrópica (densidade populacional e uso do solo) para contextualizar a degradação da qualidade da água.
    \item Calcular um Índice de Risco Ambiental (IRA) mensal através de uma abordagem de Análise por Múltiplos Critérios (AMC).
    \item **Aprimorar e implementar uma plataforma web** para a visualização interativa dos dados, permitindo a filtragem e a análise das séries temporais e espaciais do IRA e seus componentes.
    \item Disponibilizar uma ferramenta de código aberto que sirva como um modelo replicável para o monitoramento de outras bacias urbanas.
\end{itemize}
% --- FIM DA SEÇÃO OBJETIVOS ---

% --- INÍCIO DA SEÇÃO FUNDAMENTAÇÃO TEÓRICA ---
\section*{3. FUNDAMENTAÇÃO TEÓRICA}
% (Esta seção permanece a mesma da versão anterior)
\subsection{Análise por Múltiplos Critérios (AMC) para Risco Ambiental}
A Análise por Múltiplos Critérios é uma ferramenta de apoio à decisão que permite a avaliação de problemas complexos que envolvem múltiplos fatores. Neste projeto, a AMC é utilizada para integrar diferentes fontes de risco ambiental em um único indicador sintético, o Índice de Risco Ambiental (IRA). A abordagem consiste na combinação ponderada de critérios normalizados.

O IRA é um índice composto, calculado mensalmente, com a seguinte fórmula geral:
\begin{equation}
    \text{IRA} = (0.5 \times C_{QA}) + (0.3 \times C_{Pop}) + (0.2 \times C_{UCS})
\end{equation}
Onde cada critério representa: $C_{QA}$ (Qualidade da Água), $C_{Pop}$ (Pressão Populacional) e $C_{UCS}$ (Risco por Uso do Solo).
% --- FIM DA SEÇÃO FUNDAMENTAÇÃO TEÓRICA ---

% --- INÍCIO DA SEÇÃO MATERIAIS E MÉTODOS (ATUALIZADA) ---
\section*{4. MATERIAIS E MÉTODOS}
\subsection{Área de Estudo}
A Bacia do Rio Jaguaribe, localizada integralmente no perímetro urbano de João Pessoa (PB) \cite{santos2016}.

\subsection{Fontes de Dados}
\begin{itemize}
    \item \textbf{Imagens Primárias:} Sentinel-2 MSI (Nível-2A).
    \item \textbf{Dados de Apoio:} Densidade Populacional (WorldPop \cite{worldpop2020}), Uso do Solo (MapBiomas \cite{mapbiomas2021}), e Modelo Digital de Elevação (SRTM).
\end{itemize}

\subsection{Fluxo Metodológico}
O fluxo de trabalho combina o processamento de dados geoespaciais com o desenvolvimento de software.
\begin{enumerate}
    \item \textbf{Backend (Processamento de Dados):} Um script Python no GEE executa a aquisição, o pré-processamento (máscaras de nuvem e água \cite{mcfeeters1996}), o cálculo dos indicadores \cite{londe2013, nechad2010} e a aplicação da fórmula do IRA (Equação 1). Os resultados são agregados mensalmente e exportados como um arquivo CSV.
    \item \textbf{Frontend (Plataforma Web):} Uma aplicação web, já em desenvolvimento básico e hospedada na Netlify, é responsável por consumir os dados gerados. A plataforma permitirá a visualização dos dados através de camadas interativas em um mapa, gráficos de séries temporais e a aplicação de filtros (ex: por período ou por nível de risco). As tecnologias incluem HTML, CSS e JavaScript.
\end{enumerate}
% --- FIM DA SEÇÃO MATERIAIS E MÉTODOS ---

% --- INÍCIO DA SEÇÃO CRONOGRAMA (ATUALIZADA) ---
\section*{5. CRONOGRAMA}
\begin{table}[h!]
\centering
\caption{Cronograma de Execução (6 Meses).}
\label{tab:cronograma}
\begin{tabular}{ll}
\toprule
\textbf{Mês(es)} & \textbf{Atividade Principal} \\ 
\midrule
1-2 & Revisão Bibliográfica, Desenvolvimento do Script GEE e da Estrutura Base da Plataforma Web \\
3-4 & Finalização do Script, Processamento dos Dados e **Aprimoramento da Plataforma Web** \\
    & (Implementação dos filtros, gráficos dinâmicos e integração final dos dados) \\
5   & Análise Crítica dos Resultados e Redação do Artigo/Relatório Final \\
6   & Revisão, Submissão e Preparação para Apresentação/Defesa do Trabalho \\
\bottomrule
\end{tabular}
\end{table}
% --- FIM DA SEÇÃO CRONOGRAMA ---

% --- INÍCIO DA SEÇÃO RESULTADOS ESPERADOS (ATUALIZADA) ---
\section*{6. RESULTADOS ESPERADOS E PRODUTOS GERADOS}
O principal resultado esperado é uma *plataforma web de monitoramento ambiental* (\url{https://mapeamento-risco.netlify.app/}), funcional e de acesso público, que sirva como uma ferramenta de apoio à decisão para a Bacia do Rio Jaguaribe. A plataforma consolidará os seguintes produtos:
\begin{itemize}
    \item Um **sistema de visualização geoespacial** com camadas interativas para o IRA, Clorofila-a, SST, e dados de pressão antrópica.
    \item **Gráficos dinâmicos** que permitirão a análise de séries temporais e a comparação com os limites legais da legislação ambiental.
    \item A metodologia e o código-fonte (backend e frontend) disponibilizados em repositório aberto, garantindo a replicabilidade da pesquisa.
\end{itemize}
% --- FIM DA SEÇÃO RESULTADOS ESPERADOS ---

\newpage

% --- INÍCIO DA SEÇÃO REFERÊNCIAS ---
\bibliographystyle{abntex2-alf}
\bibliography{referencias} 

\end{document}