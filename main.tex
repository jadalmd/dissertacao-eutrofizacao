% ===================================================================
% TEMPLATE LATEX - PROJETO DE DISSERTAÇÃO (VERSÃO REVISADA E MELHORADA)
% Por: Orientador ABNTeX
% ===================================================================
\documentclass[12pt, a4paper]{article}

% --- PACOTES ---
\usepackage[utf8]{inputenc}
\usepackage[brazil]{babel}
\usepackage{geometry}
\usepackage{amsmath}
\usepackage{graphicx}
\usepackage{booktabs} % Para tabelas de alta qualidade (sugestão)
\usepackage[alf]{abntex2cite} % Pacote ABNT para citações (essencial)
\usepackage[hidelinks]{hyperref}

\geometry{a4paper, top=3cm, bottom=2cm, left=3cm, right=2cm} % Margens ABNT
\linespread{1.5} % Espaçamento 1.5 entre linhas

% Configurações do abntex2cite para o estilo autor-data
\begin{document}
\renewcommand{\bibname}{REFERÊNCIAS} % Renomeia a seção de bibliografia


% Início da Seção de Metadados
\begin{titlepage}
    \centering
    \vspace*{0.5cm}
    % Logo no topo da capa
    \includegraphics[width=0.25\textwidth]{logo-ppgecam.jpeg}\\[0.2cm]
    \vspace*{0.5cm}
    \textbf{\small UNIVERSIDADE FEDERAL DA PARAÍBA \\
    CENTRO DE TECNOLOGIA \\
    PROGRAMA DE PÓS-GRADUAÇÃO EM ENGENHARIA CIVIL E AMBIENTAL}
    \vspace{3cm}
    
    \textbf{\Huge Análise da Qualidade da Água e Conformidade com o Enquadramento Legal na Bacia do Rio Jaguaribe (João Pessoa-PB)}
    \vspace{3cm}
    
    \textbf{\large Jaidna Dantas de Almeida}
    \vfill
    
    \textbf{JOÃO PESSOA, PB \\ \today}
\end{titlepage}

\begin{center}
    \textbf{Área de Concentração:} Recursos Hídricos e Saneamento Ambiental \\
    \textbf{Linha de Pesquisa:} Planejamento, gestão, monitoramento e modelos matemáticos em recursos hídricos \\
    \textbf{Orientador(a):} Prof. Dr. Gerald Norbert Souza da Silva \\
    \textbf{Palavras-chave:} sensoriamento remoto, qualidade da água, enquadramento, rio Jaguaribe, Google Earth Engine.
\end{center}
% Fim da Seção de Metadados

\newpage

% Início da Seção INTRODUÇÃO
\section*{1. INTRODUÇÃO}
A gestão de recursos hídricos em bacias hidrográficas urbanas representa um dos maiores desafios contemporâneos, sendo agravada pelos efeitos das mudanças climáticas e pela contínua pressão antrópica. A bacia do rio Jaguaribe, em João Pessoa (PB), é um exemplo emblemático deste cenário, sofrendo com a degradação progressiva da qualidade de suas águas, resultado de um processo de urbanização mal planejado \cite{santos2016}.

Este quadro de degradação vai na contramão das diretrizes estabelecidas em múltiplas esferas. Em nível nacional, a Política Nacional de Recursos Hídricos (PNRH), instituída pela Lei nº 9.433/97, preconiza o uso sustentável da água \cite{pnrh1997}. Em nível estadual, a Política Estadual de Recursos Hídricos, definida pela Lei nº 6.308/96, e orientada pelo Plano Estadual de Recursos Hídricos (PERH-PB), estabelece as bases para a gestão das águas na Paraíba \cite{paraiba1996, perhPB}. O principal instrumento para garantir a qualidade da água é o Enquadramento dos Corpos de Água, regulamentado pela Resolução CONAMA nº 357/2005, que define metas de qualidade a serem alcançadas \cite{conama357}.

Contudo, a fiscalização e o monitoramento da conformidade com essas metas legais, por métodos tradicionais de coleta \textit{in situ}, são processos de alto custo e baixa frequência \cite{moraes2023}. Diante disso, o sensoriamento remoto emerge como uma tecnologia estratégica, oferecendo dados com alta frequência temporal e baixo custo operacional \cite{moraes2023}. Este projeto propõe, assim, uma metodologia automatizada, via Google Earth Engine, para avaliar a conformidade da qualidade da água do rio Jaguaribe com os padrões de sua classe de enquadramento (Classe 2), gerando um diagnóstico robusto para apoiar uma gestão ambiental alinhada com a legislação vigente.

% Fim da Seção INTRODUÇÃO

% Início da Seção OBJETIVOS
\section*{2. OBJETIVOS}
\subsection*{Objetivo Geral}
Desenvolver e aplicar uma metodologia de sensoriamento remoto para monitorar a qualidade da água da bacia do rio Jaguaribe e avaliar sua conformidade com os padrões de enquadramento da Resolução CONAMA 357/2005 para rios de Classe 2.

\subsection*{Objetivos Específicos}
\begin{itemize}
    \item Delinear programaticamente a bacia hidrográfica do Rio Jaguaribe a partir de seu exutório, utilizando algoritmos hidrológicos e um Modelo Digital de Elevação no Google Earth Engine.
    \item Calcular indicadores de qualidade da água, como Clorofila-a e Sólidos Suspensos Totais (TSS), a partir de algoritmos de sensoriamento remoto validados cientificamente.
    \item Gerar uma série temporal mensal dos indicadores para analisar a dinâmica da poluição no período.
    \item Comparar sistematicamente os resultados com os limites legais estabelecidos pela CONAMA 357/2005.
    \item Produzir mapas e gráficos que identifiquem os trechos e períodos de maior vulnerabilidade e não conformidade legal na bacia.
\end{itemize}
% Fim da Seção OBJETIVOS

% Início da Seção FUNDAMENTAÇÃO TEÓRICA
\section*{3. FUNDAMENTAÇÃO TEÓRICA}

\subsection{Enquadramento de Corpos Hídricos e a Legislação Brasileira}
O Enquadramento dos Corpos de Água é um instrumento da Política Nacional de Recursos Hídricos (Lei nº 9.433/97) que estabelece metas de qualidade a serem alcançadas ou mantidas em um corpo de água, de acordo com seus usos preponderantes. A Resolução do Conselho Nacional do Meio Ambiente (CONAMA) nº 357/2005 dispõe sobre a classificação e diretrizes ambientais para o enquadramento. Para rios de \textbf{Classe 2}, como é o caso do enquadramento do Rio Jaguaribe, destinados, entre outros, ao abastecimento para consumo humano após tratamento convencional, os principais limites de referência para este trabalho são[cite: 305]:
\begin{itemize}
    \item \textbf{Clorofila-a:} $\leq 30 \, \mu g/L$
    \item \textbf{Turbidez:} $\leq 100$ NTU (Unidade Nefelométrica de Turbidez)
\end{itemize}

\subsection{Índices de Qualidade da Água por Sensoriamento Remoto}
A estimativa de parâmetros de qualidade da água por sensoriamento remoto se baseia na resposta espectral da água, que é captada pelos sensores dos satélites. A contaminação por nutrientes, como nitrogênio e fósforo, pode levar à eutrofização, cujo primeiro sintoma é o aumento da biomassa algal, resultando no aparecimento de clorofila-a, um pigmento fotossintético indicador da degradação[cite: 206]. Para este projeto, serão utilizados os seguintes modelos:

\begin{itemize}
    \item \textbf{Clorofila-a (Chl-a):} A concentração de Clorofila-a, um indicador primário de biomassa algal e estado de eutrofização, será estimada a partir de um modelo de duas bandas. A equação a ser utilizada foi proposta e validada por \cite{londe2013} e adaptada para as bandas do sensor MSI do satélite Sentinel-2, conforme aplicado por \cite{moraes2023}:
    \begin{equation}
      Chl_a = 2.1171 + 1.68 \cdot \log_{10}\left(\frac{B_{8}}{B_{4}}\right)
    \end{equation}
    Onde:
    \begin{itemize}
        \item $Chl_a$ é a concentração de Clorofila-a em $\mu g/L$.
        \item $B_{8}$ é a refletância de superfície na banda do Infravermelho Próximo (NIR), correspondente à banda 8 do Sentinel-2[cite: 283].
        \item $B_{4}$ é a refletância de superfície na banda do Vermelho (Red), correspondente à banda 4 do Sentinel-2[cite: 283].
    \end{itemize}

    \item \textbf{Sólidos Suspensos Totais (TSS):} O TSS, que afeta diretamente a turbidez da água, será calculado com o modelo semi-analítico de \cite{nechad2010}. Este modelo é genérico e amplamente validado para diversos sensores e corpos d'água:
    \begin{equation}
     TSS = \frac{A \cdot \rho_{w}(\lambda_{Red})}{1 - (\rho_{w}(\lambda_{Red}) / C)} + B
    \end{equation}
    Para simplificação e aplicação direta, a constante B pode ser desconsiderada, resultando na forma:
    $$ TSS = \frac{A \cdot \rho_{Red}}{1 - (\rho_{Red} / C)} $$
    Onde:
    \begin{itemize}
        \item $TSS$ é a concentração de Sólidos Suspensos Totais em $g/m^3$ ou $mg/L$.
        \item $\rho_{Red}$ é a refletância da superfície da água na banda do Vermelho (Banda 4 do Sentinel-2).
        \item A e C são constantes empíricas calibradas para o sensor. Para o sensor MSI (Sentinel-2), os valores de referência são A=355.85 $g/m^3$ e C=0.1728.
    \end{itemize}
\end{itemize}
% Fim da Seção FUNDAMENTAÇÃO TEÓRICA

% Início da Seção MATERIAIS E MÉTODOS
\section*{4. MATERIAIS E MÉTODOS}

\subsection{Área de Estudo}
A pesquisa foca na bacia hidrográfica do rio Jaguaribe, um corpo hídrico que se desenvolve desde sua nascente até a foz inteiramente dentro da zona urbana do município de João Pessoa, Paraíba[cite: 36]. A área de estudo será definida de forma programática, garantindo alta precisão e reprodutibilidade. O polígono da bacia será gerado a partir do seu exutório na foz (Latitude: -7.1436, Longitude: -34.8256), utilizando como base o Modelo Digital de Elevação SRTM (Shuttle Radar Topography Mission).

\subsection{Fontes de Dados}
\begin{itemize}
    \item \textbf{Imagens de Satélite:} Serão utilizadas imagens do satélite Sentinel-2 (sensor MSI), Nível de processamento 2A (reflectância de superfície), disponíveis na plataforma Google Earth Engine. Este nível de processamento é crucial, pois já fornece dados com correção atmosférica.
    \item \textbf{Modelo Digital de Elevação (MDE):} Para o delineamento da bacia, será utilizado o MDE SRTM de 30 metros de resolução espacial (produto USGS/SRTMGL1\_003).
    \item \textbf{Padrões Legais:} As Resoluções CONAMA nº 357/2005 e nº 430/2011 servirão como base para a análise de conformidade.
\end{itemize}

\subsection{Fluxo de Trabalho Metodológico}
A metodologia será implementada em um script Python automatizado na plataforma Google Earth Engine (GEE), seguindo as etapas detalhadas abaixo, garantindo um processo fidedigno e replicável:
\begin{enumerate}
    \item \textbf{Delineamento da Bacia Hidrográfica:} Geração automática do polígono da bacia a partir do ponto de exutório definido, usando os algoritmos de hidrologia do GEE sobre o MDE SRTM.
    \item \textbf{Processamento de Dados Sentinel-2:} Filtragem da coleção de imagens por data (período de análise) e local (polígono da bacia). Será aplicada uma máscara de nuvens para remover pixels contaminados. Em seguida, será aplicada uma máscara de água robusta para isolar os pixels de interesse, utilizando a camada de classificação de cena (SCL) e o Índice de Água por Diferença Normalizada (NDWI).
    \item \textbf{Cálculo dos Indicadores:} Aplicação dos modelos matemáticos (Equações 1 e 2) sobre as imagens pré-processadas para gerar mapas mensais de Clorofila-a e Sólidos Suspensos Totais. A Turbidez será inferida a partir da forte correlação com o TSS.
    \item \textbf{Análise de Conformidade Legal:} Extração dos valores médios mensais dos indicadores para toda a área fluvial da bacia. Esses valores serão comparados diretamente com os limites estabelecidos pela CONAMA 357/2005 para rios de Classe 2.
    \item \textbf{Geração de Resultados:} Exportação dos dados agregados em formato CSV para análises estatísticas e criação de produtos visuais, como gráficos de série temporal e mapas de conformidade, que destacarão os trechos do rio e os períodos que violam os padrões legais.
\end{enumerate}
% Fim da Seção MATERIAIS E MÉTODOS

% Início da Seção CRONOGRAMA
\section*{5. CRONOGRAMA}
\begin{table}[h!]
\centering
\caption{Cronograma de Execução da Dissertação.}
\label{tab:cronograma}
\begin{tabular}{lcccc}
\toprule
\textbf{Atividade / Semestre} & \textbf{1º Sem.} & \textbf{2º Sem.} & \textbf{3º Sem.} & \textbf{4º Sem.} \\ 
\midrule
Revisão Bibliográfica & X & & & \\
Desenvolvimento do Script GEE & X & X & & \\
Processamento e Análise de Dados & & X & X & \\
Redação de Artigo & & & X & \\
Redação Final da Dissertação & & & X & X \\
Defesa & & & & X \\
\bottomrule
\end{tabular}
\end{table}
% Fim da Seção CRONOGRAMA

% Início da Seção RESULTADOS ESPERADOS
\section*{6. RESULTADOS ESPERADOS}
Espera-se como resultado principal um script Python funcional e de código aberto, capaz de realizar o monitoramento da qualidade da água e a análise de conformidade de forma automatizada na plataforma Google Earth Engine. Adicionalmente, serão gerados um relatório de dados em formato CSV, gráficos de séries temporais e mapas temáticos que identifiquem os "hotspots" de não conformidade ao longo da bacia do rio Jaguaribe. A ferramenta desenvolvida terá o potencial de subsidiar a tomada de decisão por parte dos órgãos de gestão ambiental, oferecendo um diagnóstico rápido, de baixo custo e com fundamentação científica sobre a saúde do rio.
% Fim da Seção RESULTADOS ESPERADOS

\newpage

% Início da Seção REFERÊNCIAS
\bibliographystyle{abntex2-alf}
\bibliography{referencias} % Este comando chama o arquivo 'referencias.bib'

\end{document}